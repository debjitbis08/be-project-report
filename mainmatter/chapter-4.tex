\chapter{Modular Scheduler}

Scheduling Classes have been introduced which implement an extensible hierarchy of
scheduler modules. These modules encapsulate scheduling policy
details and are handled by the scheduler core without the core
code assuming about them too much.

The introduction of scheduling classes makes the core scheduler quite extensible. These classes 
(the scheduler modules) encapsulate the scheduling policies. Each scheduler module has to implement each of
the functions suggested by the generic scheduling class.

Some Advantages of this design strategy are:
\begin{enumerate}
\item {\bfseries Easier augmentation and exclusion of scheduler policies}: This kind of design allows 
a programmer to easily introduce new scheduling policies into the kernel. The original code is modified
 to a minimal extent. This also means that testing of new scheduling algorithms become very easy.
\item {\bfseries Separate policy for real time tasks}: Real time tasks usually require special treatment by the scheduler.
A different policy for real-time tasks can be easily implemented in this kind of design.
\item {\bfseries Easier for students to understand the code}: Since MINIX 3 is widely used as an educational operating system
in many universities across the world, many of MINX 3's users are students. The design strategy used here will allow 
students to easily understand the source code and experiment with their own code.
\end{enumerate}

\section{Design Issues}

The major design issue was which functions should be removed from the core scheduler and assigned to the scheduler modules
The initial design included only the following functions in the module(these have been implemented):
\begin{itemize}
\item The enqueue function which is run each time a task becomes runnable. Based on the scheduling policy of the process a particular
enqueue funtion is called which enqueues the process in the whatever datastructure that is being used the scheduler policy.
\item The dequeue function which is run when a task is no longer runnable.
\item The pick\_proc function chooses the most appropriate process eligible to run next.
\end{itemize}
Later the following additional functions were added (these have not been implemented):
\begin{itemize}
\item The preempt\_curr is run before a task is preempted.
\item The task\_tick function is run at every clock tick.
\end{itemize}

\section{Implementation}

To implement the modular scheduler framework the following step were followed:
\begin{itemize}
\item A header \verb|sched.h| was added as suggested by the The Open Group Base Specifications Issue 6~\cite{ie04}.
The constants \verb|SCHED_FIFO|, \verb|SCHED_RR| were defined. 
\item The sturcture \verb|sched_class| was defined which encapsulates all the functions that are to be implemented
by a scheduler module i.e. the ones decided above.
\item Two new attributes were added to the process structure \verb|p_sched_policy| and \verb|p_sched_class| were the
first attribute is scheduler policy that is being used to schedule the process and the secon atrribute is a pointer
to the scheduling class.
\item Next all the scheduler related functions in the source code of the MINIX 3 kernel
was replaced by function calls to the appropiate functions in the scheduler modules. All the calls were of the form\\
\verb|proccess_pointer->sched_class->function|. The \verb|process_pointer| points to process in the context of which 
the function will be called. \verb|sched_class| is the scheduling class which the process is member of.
\end{itemize}

