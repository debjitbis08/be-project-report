\chapter{Modular Scheduler Patch}

Following is the patch to MINIX 3 version 3.1.3a which implements the modular scheduler framework:

\tiny
\begin{verbatim}

diff -uNrB /mnt/hda4p3/src/kernel/main.c /mnt/hda4p3/src_copy/kernel/main.c
--- /mnt/hda4p3/src/kernel/main.c	2007-03-30 20:47:32.000000000 +0530
+++ /mnt/hda4p3/src_copy/kernel/main.c	2008-10-25 01:27:21.000000000 +0530
@@ -71,7 +71,9 @@
 	ip = &image[i];				/* process' attributes */
 	rp = proc_addr(ip->proc_nr);		/* get process pointer */
 	ip->endpoint = rp->p_endpoint;		/* ipc endpoint */
+	rp->p_sched_class = ip->sched_class;
 	rp->p_max_priority = ip->priority;	/* max scheduling priority */
+	rp->p_sched_policy = ip->sched_policy;
 	rp->p_priority = ip->priority;		/* current priority */
 	rp->p_quantum_size = ip->quantum;	/* quantum size in ticks */
 	rp->p_ticks_left = ip->quantum;		/* current credit */
diff -uNrB /mnt/hda4p3/src/kernel/proc.c /mnt/hda4p3/src_copy/kernel/proc.c
--- /mnt/hda4p3/src/kernel/proc.c	2007-03-21 15:15:01.000000000 +0530
+++ /mnt/hda4p3/src_copy/kernel/proc.c	2008-10-25 01:43:14.000000000 +0530
@@ -44,6 +45,9 @@
 #include "proc.h"
 #include <signal.h>
 #include <minix/portio.h>
+#include "sched_classes.h"
+
+#define sched_class_highest (&rt_sched_class)
 
 /* Scheduling and message passing functions. The functions are available to 
  * other parts of the kernel through lock_...(). The lock temporarily disables 
@@ -497,32 +501,12 @@
  * The mechanism is implemented here.   The actual scheduling policy is
  * defined in sched() and pick_proc().
  */
-  int q;	 				/* scheduling queue to use */
-  int front;					/* add to front or back */
-
 #if DEBUG_SCHED_CHECK
   check_runqueues("enqueue1");
   if (rp->p_ready) kprintf("enqueue() already ready process\n");
 #endif
 
-  /* Determine where to insert to process. */
-  sched(rp, &q, &front);
-
-  /* Now add the process to the queue. */
-  if (rdy_head[q] == NIL_PROC) {		/* add to empty queue */
-      rdy_head[q] = rdy_tail[q] = rp; 		/* create a new queue */
-      rp->p_nextready = NIL_PROC;		/* mark new end */
-  } 
-  else if (front) {				/* add to head of queue */
-      rp->p_nextready = rdy_head[q];		/* chain head of queue */
-      rdy_head[q] = rp;				/* set new queue head */
-  } 
-  else {					/* add to tail of queue */
-      rdy_tail[q]->p_nextready = rp;		/* chain tail of queue */	
-      rdy_tail[q] = rp;				/* set new queue tail */
-      rp->p_nextready = NIL_PROC;		/* mark new end */
-  }
-
+  rp->p_sched_class->enqueue(rp);
   /* Now select the next process to run, if there isn't a current
    * process yet or current process isn't ready any more, or
    * it's PREEMPTIBLE.
@@ -548,10 +532,6 @@
  * it has blocked.  If the currently active process is removed, a new process
  * is picked to run by calling pick_proc().
  */
-  register int q = rp->p_priority;		/* queue to use */
-  register struct proc **xpp;			/* iterate over queue */
-  register struct proc *prev_xp;
-
   /* Side-effect for kernel: check if the task's stack still is ok? */
   if (iskernelp(rp)) { 				
 	if (*priv(rp)->s_stack_guard != STACK_GUARD)
@@ -563,23 +543,9 @@
   if (! rp->p_ready) kprintf("dequeue() already unready process\n");
 #endif
 
-  /* Now make sure that the process is not in its ready queue. Remove the 
-   * process if it is found. A process can be made unready even if it is not 
-   * running by being sent a signal that kills it.
-   */
-  prev_xp = NIL_PROC;				
-  for (xpp = &rdy_head[q]; *xpp != NIL_PROC; xpp = &(*xpp)->p_nextready) {
-
-      if (*xpp == rp) {				/* found process to remove */
-          *xpp = (*xpp)->p_nextready;		/* replace with next chain */
-          if (rp == rdy_tail[q])		/* queue tail removed */
-              rdy_tail[q] = prev_xp;		/* set new tail */
-          if (rp == proc_ptr || rp == next_ptr)	/* active process removed */
-              pick_proc();			/* pick new process to run */
-          break;
-      }
-      prev_xp = *xpp;				/* save previous in chain */
-  }
+  rp->p_sched_class->dequeue(rp);
+  if (rp == proc_ptr || rp == next_ptr)	/* active process removed */
+    pick_proc();			/* pick new process to run */
 
 #if DEBUG_SCHED_CHECK
   rp->p_ready = 0;
@@ -630,19 +596,25 @@
  * clock task can tell who to bill for system time.
  */
   register struct proc *rp;			/* process to run */
-  int q;					/* iterate over queues */
+  const struct sched_class *class;
 
   /* Check each of the scheduling queues for ready processes. The number of
    * queues is defined in proc.h, and priorities are set in the task table.
    * The lowest queue contains IDLE, which is always ready.
    */
-  for (q=0; q < NR_SCHED_QUEUES; q++) {	
-      if ( (rp = rdy_head[q]) != NIL_PROC) {
-          next_ptr = rp;			/* run process 'rp' next */
-          if (priv(rp)->s_flags & BILLABLE)	 	
-              bill_ptr = rp;			/* bill for system time */
-          return;				 
-      }
+  class = sched_class_highest; 
+  for ( ; ; ) { 
+    if ( (rp = class->pick_proc()) != NIL_PROC ) {
+      next_ptr = rp;			 /* run process 'rp' next */
+      if (priv(rp)->s_flags & BILLABLE)	 	
+        bill_ptr = rp;			/* bill for system time */
+      return;				 
+    }
+    class = class->next; 
   }
   panic("no ready process", NO_NUM);
 }
diff -uNrB /mnt/hda4p3/src/kernel/proc.h /mnt/hda4p3/src_copy/kernel/proc.h
--- /mnt/hda4p3/src/kernel/proc.h	2007-02-01 23:20:02.000000000 +0530
+++ /mnt/hda4p3/src_copy/kernel/proc.h	2008-10-25 01:22:47.000000000 +0530
@@ -21,6 +21,8 @@
   short p_rts_flags;		/* process is runnable only if zero */
   short p_misc_flags;		/* flags that do not suspend the process */
 
+  struct sched_class *p_sched_class;
+  char p_sched_policy;          /* current scheduling policy */
   char p_priority;		/* current scheduling priority */
   char p_max_priority;		/* maximum scheduling priority */
   char p_ticks_left;		/* number of scheduling ticks left */
diff -uNrB /mnt/hda4p3/src/kernel/sched.h /mnt/hda4p3/src_copy/kernel/sched.h
--- /mnt/hda4p3/src/kernel/sched.h	1970-01-01 05:30:00.000000000 +0530
+++ /mnt/hda4p3/src_copy/kernel/sched.h	2008-10-24 19:03:46.000000000 +0530
@@ -0,0 +1,26 @@
+#ifndef SCHED_H
+#define SCHED_H
+
+
+/*
+ * Scheduling policies
+ */
+#define SCHED_FIFO		1
+#define SCHED_RR		2
+#define SCHED_IDLE		3
+
+struct sched_param {
+	int sched_priority;
+};
+
+struct sched_class {
+	const struct sched_class *next;
+
+	void (*enqueue) (struct proc *rp);
+	void (*dequeue) (struct proc *rp);
+	struct proc * (*pick_proc) ();
+};
+
+
+#endif /* SCHED_H */
diff -uNrB /mnt/hda4p3/src/kernel/sched_classes.h 
          /mnt/hda4p3/src_copy/kernel/sched_classes.h
--- /mnt/hda4p3/src/kernel/sched_classes.h	1970-01-01 05:30:00.000000000 +0530
+++ /mnt/hda4p3/src_copy/kernel/sched_classes.h	2008-10-25 20:10:37.000000000 +0530
@@ -0,0 +1,8 @@
+#ifndef SCHED_CLASSES
+#define SCHED_CLASSES
+
+#include "sched_idle.c"
+#include "sched_rt.c"
+
+#endif /* SCHED_PROFILES */
diff -uNrB /mnt/hda4p3/src/kernel/sched_idle.c /mnt/hda4p3/src_copy/kernel/sched_idle.c
--- /mnt/hda4p3/src/kernel/sched_idle.c	1970-01-01 05:30:00.000000000 +0530
+++ /mnt/hda4p3/src_copy/kernel/sched_idle.c	2008-10-24 19:56:20.000000000 +0530
@@ -0,0 +1,68 @@
+/*
+ * idle-task scheduling class.
+ */
+#include "type.h"
+#include "sched.h"
+
+struct proc *rdy_head_idle; 		/* the idle queue */
+struct proc *rdy_tail_idle; 		/* the idle queue */
+
+PRIVATE _PROTOTYPE( void enqueue_idle, (struct proc *rp));
+PRIVATE _PROTOTYPE( void dequeue_idle, (struct proc *rp));
+PRIVATE _PROTOTYPE( struct proc *pick_idle, (void));
+
+PRIVATE void enqueue_idle(struct proc *rp)
+{
+  if (rdy_head_idle == NIL_PROC) {	/* add to empty queue */
+    rdy_head_idle = rdy_tail_idle = rp;	/* create a new queue */
+    rp->p_nextready = NIL_PROC;	/* mark new end */
+  }
+}
+
+PRIVATE void dequeue_idle(rp)
+register struct proc *rp;	/* this process is no longer runnable */
+{
+/* A process must be removed from the scheduling queues, for example, because
+ * it has blocked.  If the currently active process is removed, a new process
+ * is picked to run by calling pick_proc().
+ */
+  register int q = rp->p_priority;		/* queue to use */
+  register struct proc **xpp;			/* iterate over queue */
+  register struct proc *prev_xp;
+
+  /* Now make sure that the process is not in its ready queue. Remove the 
+   * process if it is found. A process can be made unready even if it is not 
+   * running by being sent a signal that kills it.
+   */
+  prev_xp = NIL_PROC;				
+  for (xpp = &rdy_head_idle; *xpp != NIL_PROC; xpp = &(*xpp)->p_nextready) {
+
+      if (*xpp == rp) {				/* found process to remove */
+          *xpp = (*xpp)->p_nextready;		/* replace with next chain */
+          if (rp == rdy_tail_idle)		/* queue tail removed */
+              rdy_tail_idle = prev_xp;		/* set new tail */
+          break;
+      }
+      prev_xp = *xpp;				/* save previous in chain */
+  }
+}
+
+PRIVATE struct proc *pick_idle()
+{
+  struct proc *rp;
+  if ( (rp = rdy_head_idle) != NIL_PROC)
+    return rp;
+  
+  panic("no ready process", NO_NUM);
+}
+
+
+/*
+ * Simple, special scheduling class for the idle task:
+ */
+static const struct sched_class idle_sched_class = 
+{ NULL, 
+  enqueue_idle, 
+  dequeue_idle, 
+  pick_idle 
+};
diff -uNrB /mnt/hda4p3/src/kernel/sched_rt.c /mnt/hda4p3/src_copy/kernel/sched_rt.c
--- /mnt/hda4p3/src/kernel/sched_rt.c	1970-01-01 05:30:00.000000000 +0530
+++ /mnt/hda4p3/src_copy/kernel/sched_rt.c	2008-10-25 20:08:27.000000000 +0530
@@ -0,0 +1,145 @@
+/*
+ * Real-Time Scheduling Class (mapped to the SCHED_FIFO and SCHED_RR
+ * policies)
+ */
+#include "type.h"
+#include "sched.h"
+
+#define MAX_RT_PRIO 15
+
+struct proc *rdy_head_rt[MAX_RT_PRIO+1]; /* the real time queues */
+struct proc *rdy_tail_rt[MAX_RT_PRIO+1]; /* the real time queues */
+
+PRIVATE _PROTOTYPE( void enqueue_rt, (struct proc *rp));
+PRIVATE _PROTOTYPE( void dequeue_rt, (struct proc *rp));
+PRIVATE _PROTOTYPE( void sched_rt, (struct proc *rp, int *queue, int *front));
+PRIVATE _PROTOTYPE( struct proc *pick_rt, (void));
+
+/*===========================================================================*
+ *                              sched_rt                                     * 
+ *===========================================================================*/
+
+PRIVATE void sched_rt(rp, queue, front)
+register struct proc *rp;			/* process to be scheduled */
+int *queue;					/* return: queue to use */
+int *front;					/* return: front or back */
+{
+/* This function determines the scheduling policy.  It is called whenever a
+ * process must be added to one of the scheduling queues to decide where to
+ * insert it.  As a side-effect the process' priority may be updated.  
+ */
+  int time_left = (rp->p_ticks_left > 0);	/* quantum fully consumed */
+
+  /* Check whether the process has time left. Otherwise give a new quantum 
+   * and lower the process' priority, unless the process already is in the 
+   * lowest queue.  
+   */
+  if (! time_left) {				/* quantum consumed ? */
+      rp->p_ticks_left = rp->p_quantum_size; 	/* give new quantum */
+      if (rp->p_priority < MAX_RT_PRIO) {  	 
+          rp->p_priority += 1;			/* lower priority */
+      }
+  }
+
+  /* If there is time left, the process is added to the front of its queue, 
+   * so that it can immediately run. The queue to use simply is always the
+   * process' current priority. 
+   */
+  *queue = rp->p_priority;
+  *front = time_left;
+}
+
+/*===========================================================================*
+ *                               enqueue_rt                                  * 
+ *===========================================================================*/
+PRIVATE void enqueue_rt(rp)
+register struct proc *rp;	/* this process is now runnable */
+{
+/* Add 'rp' to one of the queues of runnable processes.  This function is 
+ * responsible for inserting a process into one of the scheduling queues. 
+ * The mechanism is implemented here.   The actual scheduling policy is
+ * defined in sched() and pick_proc().
+ */
+  int q;	 				/* scheduling queue to use */
+  int front;					/* add to front or back */
+
+  /* Determine where to insert to process. */
+  sched_rt(rp, &q, &front);
+
+  /* Now add the process to the queue. */
+  if (rdy_head_rt[q] == NIL_PROC) {		/* add to empty queue */
+      rdy_head_rt[q] = rdy_tail_rt[q] = rp;	/* create a new queue */
+      rp->p_nextready = NIL_PROC;		/* mark new end */
+  } 
+  else if (front) {				/* add to head of queue */
+      rp->p_nextready = rdy_head_rt[q];		/* chain head of queue */
+      rdy_head_rt[q] = rp;			/* set new queue head */
+  } 
+  else {					/* add to tail of queue */
+      rdy_tail_rt[q]->p_nextready = rp;		/* chain tail of queue */	
+      rdy_tail_rt[q] = rp;			/* set new queue tail */
+      rp->p_nextready = NIL_PROC;		/* mark new end */
+  }
+}
+
+/*===========================================================================*
+ *                              dequeue_rt                                   * 
+ *===========================================================================*/
+PRIVATE void dequeue_rt(rp)
+register struct proc *rp;	/* this process is no longer runnable */
+{
+/* A process must be removed from the scheduling queues, for example, because
+ * it has blocked.  If the currently active process is removed, a new process
+ * is picked to run by calling pick_proc().
+ */
+  register int q = rp->p_priority;		/* queue to use */
+  register struct proc **xpp;			/* iterate over queue */
+  register struct proc *prev_xp;
+
+  /* Now make sure that the process is not in its ready queue. Remove the 
+   * process if it is found. A process can be made unready even if it is not 
+   * running by being sent a signal that kills it.
+   */
+  prev_xp = NIL_PROC;				
+  for (xpp = &rdy_head_rt[q]; *xpp != NIL_PROC; xpp = &(*xpp)->p_nextready) {
+
+      if (*xpp == rp) {				/* found process to remove */
+          *xpp = (*xpp)->p_nextready;		/* replace with next chain */
+          if (rp == rdy_tail_rt[q])		/* queue tail removed */
+              rdy_tail_rt[q] = prev_xp;		/* set new tail */
+          break;
+      }
+      prev_xp = *xpp;				/* save previous in chain */
+  }
+}
+
+/*===========================================================================*
+ *                                pick_rt                                    * 
+ *===========================================================================*/
+PRIVATE struct proc *pick_rt()
+{
+/* Decide who to run now.  A new process is selected by setting 'next_ptr'.
+ * When a billable process is selected, record it in 'bill_ptr', so that the 
+ * clock task can tell who to bill for system time.
+ */
+  register struct proc *rp;			/* process to run */
+  int q;					/* iterate over queues */
+
+  /* Check each of the scheduling queues for ready processes. The number of
+   * queues is defined in proc.h, and priorities are set in the task table.
+   * The lowest queue contains IDLE, which is always ready.
+   */
+  for (q=0; q < MAX_RT_PRIO; q++) {	
+      if ( (rp = rdy_head_rt[q]) != NIL_PROC) {
+          return rp;				 
+      }
+  }
+  return NIL_PROC;
+}
+
+static const struct sched_class rt_sched_class = {
+ &idle_sched_class,
+ enqueue_rt,
+ dequeue_rt,
+ pick_rt,
+};
diff -uNrB /mnt/hda4p3/src/kernel/table.c /mnt/hda4p3/src_copy/kernel/table.c
--- /mnt/hda4p3/src/kernel/table.c	2007-03-22 21:45:33.000000000 +0530
+++ /mnt/hda4p3/src_copy/kernel/table.c	2008-10-25 20:13:22.000000000 +0530
@@ -32,6 +33,8 @@
 #include "proc.h"
 #include "ipc.h"
 #include <minix/com.h>
+#include "sched.h"
+#include "sched_classes.h"
 
 /* Define stack sizes for the kernel tasks included in the system image. */
 #define NO_STACK	0
@@ -109,20 +112,20 @@
 #define no_c { 0 }, 0
 
 PUBLIC struct boot_image image[] = {
-/* process nr, pc,flags, qs,  queue, stack, traps, ipcto, call,  name */ 
-{IDLE,  idle_task,IDL_F,  8, IDLE_Q, IDL_S,     0,     0, no_c,"idle"  },
-{CLOCK,clock_task,TSK_F,  8, TASK_Q, TSK_S, TSK_T,     0, no_c,"clock" },
-{SYSTEM, sys_task,TSK_F,  8, TASK_Q, TSK_S, TSK_T,     0, no_c,"system"},
-{HARDWARE,      0,TSK_F,  8, TASK_Q, HRD_S,     0,     0, no_c,"kernel"},
-{PM_PROC_NR,    0,SRV_F, 32,      3, 0,     SRV_T, SRV_M, c(pm_c),"pm"    },
-{FS_PROC_NR,    0,SRV_F, 32,      4, 0,     SRV_T, SRV_M, c(fs_c),"vfs"   },
-{RS_PROC_NR,    0,SRV_F,  4,      3, 0,     SRV_T, SYS_M, c(rs_c),"rs"    },
-{DS_PROC_NR,    0,SRV_F,  4,      3, 0,     SRV_T, SYS_M, c(ds_c),"ds"    },
-{TTY_PROC_NR,   0,SRV_F,  4,      1, 0,     SRV_T, SYS_M,c(tty_c),"tty"   },
-{MEM_PROC_NR,   0,SRV_F,  4,      2, 0,     SRV_T, SYS_M,c(mem_c),"memory"},
-{LOG_PROC_NR,   0,SRV_F,  4,      2, 0,     SRV_T, SYS_M,c(drv_c),"log"   },
-{MFS_PROC_NR,   0,SRV_F, 32,      4, 0,     SRV_T, SRV_M, c(fs_c),"mfs"   },
-{INIT_PROC_NR,  0,USR_F,  8, USER_Q, 0,     USR_T, USR_M, no_c,"init"  },
+/* process nr, pc,flags, qs, class, policy, queue, stack, traps, ipcto, call,  name */ 
+{IDLE,  idle_task,IDL_F,  8, &idle_sched_class, SCHED_IDLE, IDLE_Q, IDL_S,
  0,     0, no_c,"idle"  },
+{CLOCK,clock_task,TSK_F,  8,   &rt_sched_class,   SCHED_RR, TASK_Q, TSK_S, 
  TSK_T,     0, no_c,"clock" },
+{SYSTEM, sys_task,TSK_F,  8,   &rt_sched_class,   SCHED_RR, TASK_Q, TSK_S, 
  TSK_T,     0, no_c,"system"},
+{HARDWARE,      0,TSK_F,  8,   &rt_sched_class,   SCHED_RR, TASK_Q, HRD_S, 
      0,     0, no_c,"kernel"},
+{PM_PROC_NR,    0,SRV_F, 32,   &rt_sched_class,   SCHED_RR,      3,     0, 
  SRV_T, SRV_M, c(pm_c),"pm"    },
+{FS_PROC_NR,    0,SRV_F, 32,   &rt_sched_class,   SCHED_RR,      4,     0, 
  SRV_T, SRV_M, c(fs_c),"vfs"   },
+{RS_PROC_NR,    0,SRV_F,  4,   &rt_sched_class,   SCHED_RR,      3,     0, 
  SRV_T, SYS_M, c(rs_c),"rs"    },
+{DS_PROC_NR,    0,SRV_F,  4,   &rt_sched_class,   SCHED_RR,      3,     0, 
  SRV_T, SYS_M, c(ds_c),"ds"    },
+{TTY_PROC_NR,   0,SRV_F,  4,   &rt_sched_class,   SCHED_RR,      1,     0, 
  SRV_T, SYS_M,c(tty_c),"tty"   },
+{MEM_PROC_NR,   0,SRV_F,  4,   &rt_sched_class,   SCHED_RR,      2,     0, 
  SRV_T, SYS_M,c(mem_c),"memory"},
+{LOG_PROC_NR,   0,SRV_F,  4,   &rt_sched_class,   SCHED_RR,      2,     0, 
  SRV_T, SYS_M,c(drv_c),"log"   },
+{MFS_PROC_NR,   0,SRV_F, 32,   &rt_sched_class,   SCHED_RR,      4,     0, 
  SRV_T, SRV_M, c(fs_c),"mfs"   },
+{INIT_PROC_NR,  0,USR_F,  8,   &rt_sched_class,   SCHED_RR, USER_Q,     0, 
  USR_T, USR_M, no_c,"init"  },
 };
 
 /* Verify the size of the system image table at compile time. Also verify that 
diff -uNrB /mnt/hda4p3/src/kernel/type.h /mnt/hda4p3/src_copy/kernel/type.h
--- /mnt/hda4p3/src/kernel/type.h	2007-02-16 21:24:28.000000000 +0530
+++ /mnt/hda4p3/src_copy/kernel/type.h	2008-10-25 01:31:18.000000000 +0530
@@ -17,6 +17,8 @@
   task_t *initial_pc;			/* start function for tasks */
   int flags;				/* process flags */
   unsigned char quantum;		/* quantum (tick count) */
+  struct sched_class *sched_class;
+  int sched_policy;
   int priority;				/* scheduling priority */
   int stksize;				/* stack size for tasks */
   short trap_mask;			/* allowed system call traps */
@@ -57,4 +59,8 @@
 
 typedef int (*irq_handler_t)(struct irq_hook *);
 
+#ifndef NIL_PROC
+#define NIL_PROC ((struct proc *) 0)
+#endif /* NIL_PROC */
+
 #endif /* TYPE_H */
\end{verbatim}

