\chapter{VTRR Scheduler Patch}

\normalsize
Following is the patch to MINIX 3 version 3.1.3a which implements the VTRR scheduler:

\tiny
\begin{verbatim}
--- src/kernel/clock.c	2008-10-28 18:48:01.000000000 +0530
+++ src_v1/kernel/clock.c	2009-05-06 11:47:31.000000000 +0530
@@ -107,9 +107,9 @@
    */ 
   if (prev_ptr->p_ticks_left <= 0 && priv(prev_ptr)->s_flags & PREEMPTIBLE) {
       if(prev_ptr->p_rts_flags == 0) {	/* if it was runnable .. */
-	lock_dequeue(prev_ptr);		/* take it off the queues */
-      	lock_enqueue(prev_ptr);		/* and reinsert it again */ 
+	scheduler_tick(prev_ptr);
       } else {
+	scheduler_tick(prev_ptr);
 	kprintf("CLOCK: %d not runnable; flags: %x\n",
 		prev_ptr->p_endpoint, prev_ptr->p_rts_flags);
       }
diff -uNrb src/kernel/main.c src_v1/kernel/main.c
--- src/kernel/main.c	2008-10-28 18:48:01.000000000 +0530
+++ src_v1/kernel/main.c	2009-05-06 16:04:57.000000000 +0530
@@ -73,6 +73,9 @@
 	ip->endpoint = rp->p_endpoint;		/* ipc endpoint */
 	rp->p_max_priority = ip->priority;	/* max scheduling priority */
 	rp->p_priority = ip->priority;		/* current priority */
+	rp->p_actual_priority = ACTUAL_PRIO(rp->p_priority);
+	rp->p_time_counter = rp->p_actual_priority;
+	rp->p_vft = 0;
 	rp->p_quantum_size = ip->quantum;	/* quantum size in ticks */
 	rp->p_ticks_left = ip->quantum;		/* current credit */
 	strncpy(rp->p_name, ip->proc_name, P_NAME_LEN); /* set process name */
@@ -168,6 +171,9 @@
    * Return to the assembly code to start running the current process. 
    */
   bill_ptr = proc_addr(IDLE);		/* it has to point somewhere */
+  qvt = 0;
+  scheduler_cycle = 0;
+  reset_scheduler();
   announce();				/* print MINIX startup banner */
   restart();
 }
diff -uNrb src/kernel/proc.c src_v1/kernel/proc.c
--- src/kernel/proc.c	2008-10-28 18:48:00.000000000 +0530
+++ src_v1/kernel/proc.c	2009-05-07 16:30:46.000000000 +0530
@@ -499,12 +499,14 @@
  */
   int q;	 				/* scheduling queue to use */
   int front;					/* add to front or back */
+  char p_time_counter;
 
 #if DEBUG_SCHED_CHECK
   check_runqueues("enqueue1");
   if (rp->p_ready) kprintf("enqueue() already ready process\n");
 #endif
 
+  rp->p_vft = MAX(qvt + (rp->p_quantum_size/rp->p_actual_priority), rp->p_vft);
   /* Determine where to insert to process. */
   sched(rp, &q, &front);
 
@@ -527,6 +529,25 @@
    * process yet or current process isn't ready any more, or
    * it's PREEMPTIBLE.
    */
+  for (rp=BEG_PROC_ADDR; rp<END_PROC_ADDR; rp++) {
+    if (! isemptyp(rp) && rp->p_rts_flags == 0 && rp != proc_addr(IDLE))
+      total_time_counter += rp->p_time_counter;
+  }
+  if (total_share <= 0) {
+    p_time_counter = rp->p_actual_priority;
+  }
+  else
+    p_time_counter = (rp->p_actual_priority*total_time_counter)/total_share; 
+
+  if (rp->p_sched_cycle == scheduler_cycle)
+    rp->p_time_counter = MIN (p_time_counter, rp->p_time_counter);
+  else
+    rp->p_time_counter = p_time_counter;
+
+  rp->p_sched_cycle = scheduler_cycle;
+
+  total_share += rp->p_actual_priority;
+
   if(!proc_ptr || proc_ptr->p_rts_flags ||
     (priv(proc_ptr)->s_flags & PREEMPTIBLE)) {
      pick_proc();
@@ -581,6 +602,9 @@
       prev_xp = *xpp;				/* save previous in chain */
   }
 
+  total_share -= rp->p_actual_priority;
+  rp->p_sched_cycle = scheduler_cycle;
+
 #if DEBUG_SCHED_CHECK
   rp->p_ready = 0;
   check_runqueues("dequeue2");
@@ -601,17 +625,6 @@
  */
   int time_left = (rp->p_ticks_left > 0);	/* quantum fully consumed */
 
-  /* Check whether the process has time left. Otherwise give a new quantum 
-   * and lower the process' priority, unless the process already is in the 
-   * lowest queue.  
-   */
-  if (! time_left) {				/* quantum consumed ? */
-      rp->p_ticks_left = rp->p_quantum_size; 	/* give new quantum */
-      if (rp->p_priority < (IDLE_Q-1)) {  	 
-          rp->p_priority += 1;			/* lower priority */
-      }
-  }
-
   /* If there is time left, the process is added to the front of its queue, 
    * so that it can immediately run. The queue to use simply is always the
    * process' current priority. 
@@ -632,18 +645,56 @@
   register struct proc *rp;			/* process to run */
   int q;					/* iterate over queues */
 
-  /* Check each of the scheduling queues for ready processes. The number of
-   * queues is defined in proc.h, and priorities are set in the task table.
-   * The lowest queue contains IDLE, which is always ready.
-   */
+  if (prev_ptr && prev_ptr->p_nextready != NIL_PROC) {
+    rp = prev_ptr->p_nextready;
+  }
+  else {
+    if (! prev_ptr || prev_ptr->p_priority >= MIN_USER_Q)
+      q = 0;
+    else
+      q = prev_ptr->p_priority+1;
+
+    for (;q<NR_SCHED_QUEUES; q++) {
+      if ( rdy_head[q] != NIL_PROC) {
+	rp = rdy_head[q];
+	break;
+      }
+    }
+  }
+  if (!prev_ptr || rp->p_priority == IDLE_Q) {
+    goto find_first;
+  }
+
+  if (rp->p_time_counter > prev_ptr->p_time_counter) {
+    next_ptr = rp;
+    if (priv(rp)->s_flags & BILLABLE)	 	
+      bill_ptr = rp;			/* bill for system time */
+    return;
+  }
+  else if (prev_ptr->p_vft - qvt - qvt_stride < 
+	   (prev_ptr->p_quantum_size/prev_ptr->p_actual_priority)) {
+    kprintf ("vft inequality\n");
+    next_ptr = rp;
+    if (priv(rp)->s_flags & BILLABLE)	 	
+      bill_ptr = rp;			/* bill for system time */
+    return;
+  }
+  else {
+    goto find_first;
+  }
+
+ find_first:
   for (q=0; q < NR_SCHED_QUEUES; q++) {	
       if ( (rp = rdy_head[q]) != NIL_PROC) {
           next_ptr = rp;			/* run process 'rp' next */
           if (priv(rp)->s_flags & BILLABLE)	 	
               bill_ptr = rp;			/* bill for system time */
+      if (rp != proc_addr(IDLE) && rp->p_time_counter == 0)
+	reset_scheduler();
           return;				 
       }
   }
+
   panic("no ready process", NO_NUM);
 }
 
@@ -667,10 +718,7 @@
       if (! isemptyp(rp)) {				/* check slot use */
 	  lock(5,"balance_queues");
 	  if (rp->p_priority > rp->p_max_priority) {	/* update priority? */
-	      if (rp->p_rts_flags == 0) dequeue(rp);	/* take off queue */
 	      ticks_added += rp->p_quantum_size;	/* do accounting */
-	      rp->p_priority -= 1;			/* raise priority */
-	      if (rp->p_rts_flags == 0) enqueue(rp);	/* put on queue */
 	  }
 	  else {
 	      ticks_added += rp->p_quantum_size - rp->p_ticks_left;
@@ -691,6 +739,56 @@
 }
 
 /*===========================================================================*
+ *				reset_scheduler				     *
+ *===========================================================================*/
+PUBLIC void reset_scheduler()
+{
+  register struct proc *rp;
+
+ /*kprintf ("Resetting scheduler(%d,%d)\n", total_share, total_time_counter);*/
+
+  total_share = 0;
+  qvt = 0;
+  for (rp=BEG_PROC_ADDR; rp<END_PROC_ADDR; rp++) {
+    if (! isemptyp(rp) && rp->p_rts_flags == 0 && rp != proc_addr(IDLE)) {
+      lock(6, "reset_scheduler");
+      rp->p_time_counter = rp->p_actual_priority;
+      total_share += rp->p_actual_priority;
+    }
+    unlock(6);
+  }
+  if (total_share > 0)
+    qvt_stride = Q_SIZE/total_share;
+  else
+    qvt_stride = 0;
+
+  scheduler_cycle = (scheduler_cycle < MAX_SCHED_CYCLE) ? 
+    scheduler_cycle++ : 0;
+
+  pick_proc();
+
+}
+
+/*===========================================================================*
+ *				scheduler_tick				     *
+ *===========================================================================*/
+PUBLIC void scheduler_tick(rp)
+register struct proc *rp;
+{
+  int time_left = (rp->p_ticks_left > 0);	/* quantum fully consumed */
+
+  if (total_share > 0)
+    qvt += rp->p_quantum_size/total_share;
+
+  rp->p_time_counter--;
+
+  if (! time_left) {				/* quantum consumed ? */
+      rp->p_ticks_left = rp->p_quantum_size; 	/* give new quantum */
+  }
+  rp->p_vft += rp->p_quantum_size/rp->p_actual_priority;
+}
+
+/*===========================================================================*
  *				lock_send				     *
  *===========================================================================*/
 PUBLIC int lock_send(dst_e, m_ptr)
diff -uNrb src/kernel/proc.h src_v1/kernel/proc.h
--- src/kernel/proc.h	2008-10-28 18:48:01.000000000 +0530
+++ src_v1/kernel/proc.h	2009-05-07 11:28:32.000000000 +0530
@@ -22,10 +22,15 @@
   short p_misc_flags;		/* flags that do not suspend the process */
 
   char p_priority;		/* current scheduling priority */
+  char p_actual_priority;
+  char p_time_counter;
   char p_max_priority;		/* maximum scheduling priority */
   char p_ticks_left;		/* number of scheduling ticks left */
   char p_quantum_size;		/* quantum size in ticks */
 
+  long int p_vft;
+  int p_sched_cycle;
+
   struct mem_map p_memmap[NR_LOCAL_SEGS];   /* memory map (T, D, S) */
 
   clock_t p_user_time;		/* user time in ticks */
@@ -120,6 +125,10 @@
 #define MIN_USER_Q	  14	/* minimum priority for user processes */
 #define IDLE_Q		  15    /* lowest, only IDLE process goes here */
 
+#define MAX_SCHED_CYCLE 100
+#define Q_SIZE 8
+#define ACTUAL_PRIO(p) (IDLE_Q - p + 1)
+
 /* Magic process table addresses. */
 #define BEG_PROC_ADDR (&proc[0])
 #define BEG_USER_ADDR (&proc[NR_TASKS])
@@ -149,4 +158,9 @@
 EXTERN struct proc *rdy_head[NR_SCHED_QUEUES]; /* ptrs to ready list headers */
 EXTERN struct proc *rdy_tail[NR_SCHED_QUEUES]; /* ptrs to ready list tails */
 
+long int qvt;
+long int qvt_stride;
+long int total_share;
+int scheduler_cycle;
 #endif /* PROC_H */
diff -uNrb src/kernel/proto.h src_v1/kernel/proto.h
--- src/kernel/proto.h	2008-10-28 18:48:01.000000000 +0530
+++ src_v1/kernel/proto.h	2009-05-06 01:28:39.000000000 +0530
@@ -33,6 +33,8 @@
 _PROTOTYPE( void lock_enqueue, (struct proc *rp)			);
 _PROTOTYPE( void lock_dequeue, (struct proc *rp)			);
 _PROTOTYPE( void balance_queues, (struct timer *tp)			);
+_PROTOTYPE( void reset_scheduler, (void)			);
+_PROTOTYPE( void scheduler_tick, (struct proc *rp)			);
 #if DEBUG_ENABLE_IPC_WARNINGS
 _PROTOTYPE( int isokendpt_f, (char *file, int line, endpoint_t e, int *p, int f));
 #define isokendpt_d(e, p, f) isokendpt_f(__FILE__, __LINE__, (e), (p), (f))
diff -uNrb src/kernel/system/do_exec.c src_v1/kernel/system/do_exec.c
--- src/kernel/system/do_exec.c	2008-10-28 18:47:58.000000000 +0530
+++ src_v1/kernel/system/do_exec.c	2009-05-06 01:31:34.000000000 +0530
@@ -51,7 +51,8 @@
   } else {
   	strncpy(rp->p_name, "<unset>", P_NAME_LEN);
   }
-  
+  rp->p_time_counter = rp->p_actual_priority;
+  rp->p_vft = 0;
   return(OK);
 }
 #endif /* USE_EXEC */
diff -uNrb src/kernel/system/do_fork.c src_v1/kernel/system/do_fork.c
--- src/kernel/system/do_fork.c	2008-10-28 18:47:58.000000000 +0530
+++ src_v1/kernel/system/do_fork.c	2009-05-06 01:48:15.000000000 +0530
@@ -55,6 +55,9 @@
   rpc->p_user_time = 0;		/* set all the accounting times to 0 */
   rpc->p_sys_time = 0;
 
+  rpc->p_time_counter = rpc->p_actual_priority;
+  rpc->p_vft = 0;
+
   /* Parent and child have to share the quantum that the forked process had,
    * so that queued processes do not have to wait longer because of the fork.
    * If the time left is odd, the child gets an extra tick.
diff -uNrb src/kernel/system/do_nice.c src_v1/kernel/system/do_nice.c
--- src/kernel/system/do_nice.c	2008-10-28 18:47:59.000000000 +0530
+++ src_v1/kernel/system/do_nice.c	2009-05-06 01:31:25.000000000 +0530
@@ -48,6 +48,7 @@
        */
       RTS_LOCK_SET(rp, NO_PRIORITY);
       rp->p_max_priority = rp->p_priority = new_q;
+      rp->p_actual_priority = ACTUAL_PRIO(new_q);
       RTS_LOCK_UNSET(rp, NO_PRIORITY);
 
       return(OK);
diff -uNrb src/kernel/table.c src_v1/kernel/table.c
--- src/kernel/table.c	2008-10-28 18:48:01.000000000 +0530
+++ src_v1/kernel/table.c	2009-05-06 04:26:06.000000000 +0530
@@ -110,19 +110,19 @@
 
 PUBLIC struct boot_image image[] = {
 /* process nr, pc,flags, qs,  queue, stack, traps, ipcto, call,  name */ 
-{IDLE,  idle_task,IDL_F,  8, IDLE_Q, IDL_S,     0,     0, no_c,"idle"  },
-{CLOCK,clock_task,TSK_F,  8, TASK_Q, TSK_S, TSK_T,     0, no_c,"clock" },
-{SYSTEM, sys_task,TSK_F,  8, TASK_Q, TSK_S, TSK_T,     0, no_c,"system"},
-{HARDWARE,      0,TSK_F,  8, TASK_Q, HRD_S,     0,     0, no_c,"kernel"},
-{PM_PROC_NR,    0,SRV_F, 32,      3, 0,     SRV_T, SRV_M, c(pm_c),"pm"    },
-{FS_PROC_NR,    0,SRV_F, 32,      4, 0,     SRV_T, SRV_M, c(fs_c),"vfs"   },
-{RS_PROC_NR,    0,SRV_F,  4,      3, 0,     SRV_T, SYS_M, c(rs_c),"rs"    },
-{DS_PROC_NR,    0,SRV_F,  4,      3, 0,     SRV_T, SYS_M, c(ds_c),"ds"    },
-{TTY_PROC_NR,   0,SRV_F,  4,      1, 0,     SRV_T, SYS_M,c(tty_c),"tty"   },
-{MEM_PROC_NR,   0,SRV_F,  4,      2, 0,     SRV_T, SYS_M,c(mem_c),"memory"},
-{LOG_PROC_NR,   0,SRV_F,  4,      2, 0,     SRV_T, SYS_M,c(drv_c),"log"   },
-{MFS_PROC_NR,   0,SRV_F, 32,      4, 0,     SRV_T, SRV_M, c(fs_c),"mfs"   },
-{INIT_PROC_NR,  0,USR_F,  8, USER_Q, 0,     USR_T, USR_M, no_c,"init"  },
+{IDLE,  idle_task,IDL_F,  Q_SIZE, IDLE_Q, IDL_S,     0,     0, no_c,"idle"  },
+{CLOCK,clock_task,TSK_F,  Q_SIZE, TASK_Q, TSK_S, TSK_T,     0, no_c,"clock" },
+{SYSTEM, sys_task,TSK_F,  Q_SIZE, TASK_Q, TSK_S, TSK_T,     0, no_c,"system"},
+{HARDWARE,      0,TSK_F,  Q_SIZE, TASK_Q, HRD_S,     0,     0, no_c,"kernel"},
+{PM_PROC_NR,    0,SRV_F,  Q_SIZE,      3, 0,     SRV_T, SRV_M, c(pm_c),"pm"  },
+{FS_PROC_NR,    0,SRV_F,  Q_SIZE,      4, 0,     SRV_T, SRV_M, c(fs_c),"vfs" },
+{RS_PROC_NR,    0,SRV_F,  Q_SIZE,      3, 0,     SRV_T, SYS_M, c(rs_c),"rs"  },
+{DS_PROC_NR,    0,SRV_F,  Q_SIZE,      3, 0,     SRV_T, SYS_M, c(ds_c),"ds"  },
+{TTY_PROC_NR,   0,SRV_F, Q_SIZE,      1, 0,     SRV_T, SYS_M,c(tty_c),"tty"  },
+{MEM_PROC_NR,   0,SRV_F,Q_SIZE,      2, 0,     SRV_T, SYS_M,c(mem_c),"memory"},
+{LOG_PROC_NR,   0,SRV_F,  Q_SIZE,      2, 0,     SRV_T, SYS_M,c(drv_c),"log" },
+{MFS_PROC_NR,   0,SRV_F, Q_SIZE,      4, 0,     SRV_T, SRV_M, c(fs_c),"mfs" },
+{INIT_PROC_NR,  0,USR_F,  Q_SIZE, USER_Q, 0,     USR_T, USR_M, no_c,"init"  },
 };
 
 /* Verify the size of the system image table at compile time. Also verify that
 
\end{verbatim}

