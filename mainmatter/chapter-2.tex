\chapter{Background}

Proportional share resource management provides a flexible and useful abstraction for multiplexing a scarce resource among user and applications. The basic idea is each client has an associated weight,
and resources are allocated to the clients in proportion to their respective weights. Because of its usefulness, 
many proportional share scheduling mechanisms have been developed.

Proportional share scheduling mechanisms were first developed decades ago with the introduction of
weighted round-robin scheduling. Later, fair-share algorithms based on controlling priority
values were developed and incorporated into some UNIX based operating systems.
These earlier mechanisms were typically fast, requiring only constant time to select a client for execution.
However, they were limited in the accuracy with which they could achieve proportional sharing.
As a result, starting in the late 1980s, fair queueing algorithms were developed first for
network packet scheduling and later for CPU scheduling. These algorithms provided better
proportional sharing accuracy.
                                               
Previous proportional sharing mechanisms can be
classified into four categories: those that are fast but
have weaker proportional fairness guarantees, those that
map well to existing scheduler frameworks in current
commercial operating systems but have no well-defined
proportional fairness guarantees, those that have strong
overhead, and those that have weaker proportional fairness guarantees but have higher scheduling overhead.
The four categories correspond to round-robin, fair-share, fair queueing, and lottery mechanisms.

\section{Fair-Share}

Fair-share schedulers arose as a result of
a need to provide proportional sharing among users in a
way compatible with a UNIX-style time-sharing framework. In UNIX time-sharing, scheduling is done based
on multi-level feedback with a set of priority queues.
Each client has a priority which is adjusted as it executes.
The scheduler executes the client with the highest priority. The idea of fair-share was to provide proportional
sharing among users by adjusting the priorities of a user's clients suitably. Fair-share provides proportional sharing by effectively running clients at different frequencies, as opposed to WRR which only adjusts
the size of the clients’ time quanta. Fair-share schedulers were compatible with UNIX scheduling frameworks and relatively easy to deploy in existing UNIX
environments. Unlike round-robin scheduling, the focus was on providing proportional sharing to groups of
users as opposed to individual clients. However, the approaches were often ad-hoc and it is difficult to formalize the proportional fairness guarantees they provided.
Empirical measurements show that these approaches
only provide reasonable proportional fairness over relatively large time intervals. It is almost certainly the
case that the allocation errors in these approaches can be
very large.

\section{Proportional Share Fairness}

We use the same definition of fairness as Caprita et al. \cite{ccn05} which is based on the
Generalized Processor Sharing model \cite{kl76}.

The idea is that there are multiple clients requesting access to a shared resource. Further,
these clients have different privileges which are expressed as weights. The weights define
how much service a client should receive in relation to other clients and how large its share
of the resource should be. For example, a client with a weight of 2 should receive twice as
much service as a client with a weight of 1. Further, in case the sum of all weights is 10, it
should receive 20\% of the available resource.

Formally, this can be expressed as follows: Let $W_{C_a}(t_1, t_2)$ be the amount of service received, 
i.e. execution time, by a client $C_a$ in the interval between $t_1$ and $t_2$. Given
many arbitrary clients, $C_j$ and their associated positive weights $\phi_j$ a scheduling algorithm
is fair if and only if

\begin{equation}
  \forall a. W_{C_a}(t_1, t_2) = (t_2 - t_1)\frac{\phi_a}{\sum_j\phi_j} \label{1}
\end{equation}

is true for all intervals [$t_1$ , $t_2$].

This means that each client should receive service exactly \emph{proportional} to its
\emph{share} of the total weight.

\paragraph{Service Error}

Obviously, no real world scheduling algorithm can be entirely fair. This is due to hardware
and efficiency constraints that make it necessary to allocate the CPU in discrete time units
(TUs). As only one client can advance at one point in time, a client gets ahead of its
proportional share of service while running and falls behind while being queued, thus
creating an unfair state. The difference between what a client should have received in an
interval and what it actually received is called the \emph{service error}.

Let $W_{C_a}(t_1, t_2)$ denote the work actually received by $C_a$ in $[t_1 , t_2]$.
Then the service error $E_{C_a}(t_1 , t_2 )$ of a client $C_a$ is defined as:

\begin{equation}
  E_{C_1}(t_1, t_2) = W_{C_a}(t_1, t_2) - (t_2 - t_1)\frac{\phi_a}{\sum_j\phi_j}
\end{equation}
                                             
A negative service error indicates that a client has fallen behind, while a positive service
error signals disproportionate resource consumption. Note that while it may not be
 possible to avoid service errors at all times, it is at least possible to achieve perfect fairness every
$\sum_j \phi_j$ rounds. Therefore it must be the goal of real world proportional share schedulers to
minimise the maximum absolute service errors in between. Ideally it should be possible
to give constant bounds for the maximum and minimum service errors for a scheduling
algorithm, i.e. it should have an \emph{O}(1) service error.

\paragraph{Virtual Time}

When deciding which client should receive service next, a proportional share scheduler
has to consider past resource allocation. Ideally, given two clients $C_a$ and $C_b$, we would
like to simply compare their received service times $W_{C_a}(t_1 , t_2)$ and $W_{C_b}(t_1 , t_2)$ and select
the one which received less service so far. But doing so in terms of the absolute number of
allocated time units is pointless, as clients with larger weights should receive more service
than clients with smaller weights.

A measure for a client's normalised progress can be obtained by first scaling the received
service by the weight to account for different rights to the resource. The result is called
\emph{virtual time} \cite{dks89,nvz01} and is defined as follows:

\begin{equation}
  VT_{C_a}(t) = \frac{W_{C_a}(0 , t)}{\phi_a}
\end{equation}

The virtual times of clients can be used to compare their past resource allocation directly, as differences in weights allow for proportional differences in absolute service times.

\paragraph{Queue Virtual Time}

While virtual time allows us to tell whether a client is ahead of another client easily, it
does not tell us whether these clients are ahead or behind of their proportional share, i.e.
whether they have a positive or negative service error. Starting from a fair state, we can
derive a term that allows for such a comparison.

In a fair state, equation\eqref{1}  holds. Assuming $t_0$ = 0, the equation can be rewritten as
follows:

\begin{align}
                 &\forall  a. W_{C_a}(0, t) &= &\quad t*\frac{\phi_a}{\sum_j\phi_j} \nonumber\\
\Leftrightarrow  &\forall  a. \frac{W_{C_a}(0, t)}{\phi_a} &= &\quad t*\frac{1}{\sum_j\phi_j} \nonumber\\
\Leftrightarrow  &\forall  a. VT_{C_a}(t)  &= &\quad \frac{t}{\sum_j\phi_j}\nonumber
\end{align}

The term on the right side of the last equation is independent of the client weight $\phi_a$. This
shows on the one hand that in a fair state all virtual times must be equal and on the other
hand gives us an easy to compute figure allowing for quick comparisons of clients virtual
time. This is called \emph{queue virtual time}~\cite{nvz01}.

\begin{equation}
  QVT(t) = \frac{t}{\sum_j\phi_j}
\end{equation}

As the virtual time of a client should always be equal to the queue virtual time, we can
infer from a smaller virtual time that the client has fallen behind its proportional allocation
and vice versa for larger virtual times.


\section{Defining Interactivity and Reponsiveness}

These two terms, usually used interchangeably in operating system literature are difficult to define quantities.
They are heavily dependent on the perception of the user and are difficult to measure (if one cannot define 
something how can it be measured).

We would prefer to define these quantities as:
\begin{description}
\item[Responsiveness] is the rate at which the processes can proceed under different kind of loads.
\item[Interactivity] is the scheduler latency that a process experiences under different load conditions
\end{description}

The two definitions are very close, but there is a subtle difference. Responsiveness will tend to measure the long-term
behaviour of a process, whereas interactivity will measure the short-term behaviour or the `snappiness'.

